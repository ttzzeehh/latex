\documentclass{sxtzh}
\begin{document}
    \tzhshowanswer{教师版详答}
    \danxuan
\begin{enumerate}
    \item $ABCDEFGHIGKLMNOPQRSTUVWXYZ$,\par
    ABCDEFGHIGKLMNOPQRSTUVWXYZ\par
    $abcdefghigklmnopqrstuvwxyz$\par
    abcdefghigklmnopqrstuvwxyz\par
        $\alpha \beta \gamma \delta \epsilon \zeta \eta \theta \iota \kappa \lambda \mu \nu \xi \omicron \pi \rho \sigma \tau \upsilon \phi \chi \psi \omega$
        $\frac{1}{2}+\frac{x}{y}=\frac{3}{2}=\frac{3}{2}$这是事实上生活时尚\par
        $\dfrac{1}{2}+\dfrac{2}{3}=\dfrac{x}{y}$
    \item \tiyuan{成都七中$23$届高三上$10$月月考T$16$}$\triangle APB, \vartriangle APB$辅助角公式是我国清代数学家李善兰发现的用来化简三角函数的一个公式,
    其内容为$a \sin{x} +b \cos {x} =\sqrt{a^2+b^2}\sin(x+\varphi)$.(其中$a \neq 0,b\in R, \tan \varphi = \dfrac{b}{a}$).
    已知函数$f(x)=\sin{\omega x} + m\cos{\omega x} (m>0,\omega>0)$的图像的两个相邻零点之间的距离小于$\uppi$,$x=\dfrac{\uppi}{6}$为函数
    $f(x)$的极大值点,且$f\left(\dfrac{\uppi}{3}\right)=\sqrt{3}$,则实数$\omega $的最小值为
    \daan{$13$}
    \begin{solution}
        $f^\prime (x) = \omega \cos \omega x - m \omega \sin \omega x$,
        \par
        $\begin{dcases}
            f^\prime\left(\dfrac{\uppi}{6}\right)=0 \\
            f\left(\dfrac{\uppi}{3}\right)=0 
        \end{dcases}$
        $\Rightarrow$
        $\begin{dcases}
         m\sin{\dfrac{\uppi}{6}\omega} = \cos{\dfrac{\uppi}{6}\omega }\\
           \sin{\dfrac{\uppi}{3}\omega} + m\cos{\dfrac{\uppi}{3}\omega}=\sqrt{3} 
        \end{dcases}$
        $\Rightarrow$
       $\begin{dcases}
        m\sin \dfrac{\uppi}{6}\omega = \cos \dfrac{\uppi}{6}\omega \\
        2\sin{\dfrac{\uppi}{6}\omega} \cdot \cos{\dfrac{\uppi}{6}\omega}+2m\cos^2{\dfrac{\uppi}{6}\omega}=m+\sqrt{3}
       \end{dcases}$
       \par
       可得$m=\sqrt{3}$,$f(x)=2\sin(\omega x+\dfrac{\uppi}{3})$
       \par
       因为$T < 2\uppi,\Rightarrow \dfrac{2\uppi}{\omega}<2\uppi,\Rightarrow \omega >1$
       又因为$x = \dfrac{\uppi}{6}$为$f(x)$的极大值点,$\dfrac{\uppi}{6} \omega  + \dfrac{\uppi}{3} = \dfrac{\uppi}{2} + 2k\uppi,k\in Z$
       所以$\omega = 12k + 1,k\in Z$,即$\omega_{min} = 13$
    \end{solution}
    \item \tiyuan{四川卷} 这是第一题
    \daan{选A}
    \begin{solution}
        为什么选择A
        \begin{fangfa}
            方法一
        \end{fangfa}
        \begin{fangfa}
            方法二
        \end{fangfa}
    \end{solution}
    \item 这是第二题
    \daan{选B}
    \begin{solution}
        为什么选择B
    \end{solution}
    \item 这是第三题
    \daan{选C}
    \begin{solution}
        为什么选择C
    \end{solution}
\end{enumerate}
\tiankong
\begin{enumerate}[resume]
    \item 这是第二题
    \daan{选B}
    \begin{solution}
        为什么选择B
    \end{solution}
    \item 这是第三题
    \daan{选C}
    \begin{solution}
        为什么选择C
    \end{solution}
\end{enumerate}
\jieda
\begin{enumerate}[resume]
    \item 这是第二题
    \daan{选B}
    \begin{solution}
        为什么选择B
    \end{solution}
    \item 这是第三题
    \daan{选C}
    \begin{solution}
        为什么选择C
    \end{solution}
    \item 这是第二题
    \daan{选B}
    \begin{solution}
        为什么选择B
    \end{solution}
    \item 这是第三题
    \daan{选C}
    \begin{solution}
        为什么选择C
    \end{solution}
    \item 这是第二题
    \daan{选B}
    \begin{solution}
        为什么选择B
    \end{solution}
    \item 这是第三题
    \daan{选C}
    \begin{solution}
        为什么选择C
    \end{solution}
    \item 这是第二题
    \daan{选B}
    \begin{solution}
        为什么选择B
    \end{solution}
    \item 这是第三题
    \daan{选C}
    \begin{solution}
        为什么选择C
    \end{solution}
    \item 这是第三题
    \daan{选C}
    \begin{solution}
        为什么选择C
    \end{solution}
    \item 这是第二题
    \daan{选B}
    \begin{solution}
        为什么选择B
    \end{solution}
    \item 这是第三题
    \daan{选C}
    \begin{solution}
        为什么选择C
    \end{solution}
    \item 这是第三题
    \daan{选C}
    \begin{solution}
        为什么选择C
    \end{solution}
    \item 这是第二题
    \daan{选B}
    \begin{solution}
        为什么选择B
    \end{solution}
    \item 这是第三题
    \daan{选C}
    \begin{solution}
        为什么选择C
    \end{solution}
    \item 这是第三题
    \daan{选C}
    \begin{solution}
        为什么选择C
    \end{solution}
    \item 这是第二题
    \daan{选B}
    \begin{solution}
        为什么选择B
    \end{solution}
    \item 这是第三题
    \daan{选C}
    \begin{solution}
        为什么选择C
    \end{solution}
\end{enumerate}
\end{document}